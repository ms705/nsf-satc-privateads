\section*{Data Management Plan}

% Plans for data management and sharing of the products of
% research. Proposals must include a supplementary document of no more
% than two pages labeled “Data Management Plan”. This supplement should
% describe how the proposal will conform to NSF policy on the dissemination
% and sharing of research results (see AAG Chapter VI.D.4), and may include:
%
% - the types of data, samples, physical collections, software,
%   curriculum materials, and other materials to be produced in the course
%   of the project;
%
% - the standards to be used for data and metadata format and content
%   (where existing standards are absent or deemed inadequate, this should
%   be documented along with any proposed solutions or remedies);
%
% - policies for access and sharing including provisions for appropriate
%   protection of privacy, confidentiality, security, intellectual property,
%   or other rights or requirements;
%
% - policies and provisions for re-use, re-distribution, and the
%   production of derivatives; and
%
% - plans for archiving data, samples, and other research products,
%   and for preservation of access to them.

%
The goal of this project is to explore the cryptographic and systems software
design of a privacy-preserving online ad ecosystem.
%
As part of this effort, we plan to build working prototypes of a
privacy-preserving ad platform, the necessary client browser integration, and
the cryptographic libraries needed to realize the blind signatures our design
is based on.
%
All source code, configuration files, and instructions for reproducing published
measurements of our systems will be made public, hosted via GitHub version
control and accessible from the project website at Brown University.
%
The data generated through the work described in this proposal will consist of
papers, source code,  benchmarks, and experimental results.
%

%
The PIs bring extensive experience with building and contributing to open-source
software.
%
PI Schwarzkopf's prior systems have become popular open-source projects (\eg the
Noria dataflow system, which has 23 contributors, 188 pull requests, and 4,400 stars
on GitHub), or have been integrated with popular open-source software (\eg the
Firmament cluster scheduler, for which an open-source community developed a
plugin for Google's Kubernetes cluster manager).
%
All of this software is available for public access on GitHub.
%

\paragraph{Policies for Data Access and Sharing.}
We will make the source code for our cryptographic protocols, as well as any
browser modifications and prototypes server-side software available for public
download under open-source licenses.
%
To ensure broad impact of our work, the source code will be available for reuse
without restriction.
%

\paragraph{Reproducibility of Experimental Results}
%
We will publish our results in peer-reviewed security and systems venues such as
CRYPTO, IEEE S\&P, CCS, USENIX Security, SOSP, OSDI, or NSDI.
%
When a venue has a pre-publiction artifact evaluation process, we will
participate and support third-party reproduction of our results.
%

\paragraph{Data Formats.}
All data generated will be labeled and stored in digital formats, and we will
ensure that any empirical data produced in our experiments and our source code
distributions use standard formats.
%
For papers, we will keep both a PDF copy of the document and the {\LaTeX}
source.
%
Experimental results will be stored in standard formats compatible with
widely-used software tools, such as spreadsheets or ASCII text files.
%
We will also include sufficient documentation with our released code and data
to allow third parties to reconstruct our published results.
%

\paragraph{Data Retention.}
Data will be retained for at least three years beyond the award period, as
required by NSF guidelines.
%
The GitHub version control repositories will ensure the integrity of data and
allow for long-term retention.
%

\paragraph{Data Privacy.}
This research project does not involve confidential data or source code.
