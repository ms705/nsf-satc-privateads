% If any PI or co-PI identified on the project has received NSF funding in
% the past five years, information on the award(s) is required. Each PI
% and co-PI who has received more than one award(excluding amendments)
% must report on the award most closely related to the proposal. The
% following information must be provided:
%
% (a) the NSF award number, amount and period of support;
%
% (b) the title of the project;
%
% (c) a summary of the results of the completed work, including, for a
% research project, any contribution to the development of human resources
% in science and engineering;
%
% (d) publications resulting from the NSF award;
%
% (e) a brief description of available data, samples, physical collections
% and other related research products not described elsewhere; and
%
% (f) if the proposal is for renewed support, a description of the relation
% of the completed work to the proposed work.

\section{Results From Prior NSF Support}
\label{s:prior}

\noindent\textbf{Lysyanskaya} is supported by:
\textbf{SaTC: CORE: Small: Mercurial Signatures and Applications to Privacy-Preserving Authentication}
(CNS-2154170, July 2022--June 2025, \$499,930).
\textit{Results:} None yet, project just started. \textit{Intellectual merit}:  The goal of this project is to design and analyze new mercurial signatures with the aim of bridging the conceptual, security, and use-case disconnects of anonymous credentials to reduce the barriers to adoption of privacy-enhancing authentication services.
\textit{Broader Impacts:} Making anonymous authentication more accessible and efficient; diversity and educational impacts.  
\textbf{PIPP Phase I: Mobility Analysis for Pandemic Prevention Strategies (MAPPS)}
(CBET-2154941, August 2022--January 2024, \$999,211).
\textit{Results:} None yet, project just started. \textit{Intellectual merit}: Studying human mobility data in connection with the spread of disease, in a way that's ethical and respects privacy.
\textit{Broader Impacts:} Blunting the effect of pandemics.  

\textbf{Schwarzkopf} has received the following funding:
%
\textbf{CAREER: Privacy-Compliant Web Services By Construction} (CNS-2045170, February 2021--January 2026, \$585,025).
%
\emph{Results:} a new database system that simplifies privacy compliance
(paper under review).
%
\emph{Intellectual Merit:} design of web services that comply with privacy
laws (\eg the GDPR) by construction, new database system design that makes
privacy compliance easy.
%
\emph{Broader Impact:} creating open-source, off-the shelf software
(database, libraries) that help organizations do a better job at privacy
compliance.
%
\textbf{EAGER: SaTC-EDU: Instilling a Mindset of Adversarial Thinking into
Computer Science Courses Early and Often} (DGE-2039354,
September 2020--August 2022, \$297,881, joint with Kathryn Fisler, Shiram
Krishnamurthi, Tim Nelson, Stephen Bach).
%
\emph{Results:} insights into teaching students adversarial thinking and
systems that force developers to think about adversarial inputs; papers
published in ICER 2021, SIGCSE 2021, and SIGMOD 2021.
%
\emph{Intellectual Merit:} investigates the effect of teaching adversarial
thinking skills to students.
%
\emph{Broader Impact:} insight into curriculum and assignment design
around adversarial thinking, reusable frameworks and assignments.
%
