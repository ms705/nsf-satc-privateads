\section{Background and Problem Areas}
\label{s:bg}

\subsection{Ad Targeting}

%
In a \emph{targeted advertising} setting, a publisher shares information about
their users with the platform, and the platform chooses ads to show to the
users from their inventory of advertiser-provided ads.
%

\begin{itemize}
  \item RTB (real-time bidding)
  \item Need high-fidelity information about user interests
  \item Cross-site trackers, reference Apple privacy change impact
  \item Google FLOC?
\end{itemize}

\subsection{Paying for Ads and Preventing Abuse}
\todo{there are several types of fraud we want to address:
(1) is the person who clicked on an ad legitimate, not a sybil?  
(2) correctly accounting for which ad led a (legitimate) user to later visit the advertiser's site and/or complete a purchase. (3) making sure that the publisher/platform are rewarded for showing a successful ad.
}

%\subsection{A better approach: Privacy-First Advertising}

%Our starting point is simple: we seek the reduce the information exchanged
%between the parties in the ad ecosystem to \emph{the minimum information
%necessary for successful ad targeting and for the parties to trust each other}
%that they have executed their part of the protocol faithfully and correctly.
%

%

\subsection{Blind Signatures and Anonymous Credentials}
\label{s:bg-crypto}

\paragraph{Blind signatures.} A blind signing protocol is a protocol between a signer and a user in which the user outputs a digital signature on the desired message, while the signer learns nothing about the message or the resulting signature.
A blind signature scheme is a signature scheme that has a blind signing protocol.  In spite of having a relatively long history (they were introduced almost forty years ago by David Chaum~\cite{C:Chaum82}), blind signatures are a subject of excitement in the cryptography research community at the moment because they can be used as privacy-preserving authentication tokens that can replace browser cookies in certain applications, for example by the VPN by Google One~\cite{google-one-vpn} and Apple's iCloud Private Relay~\cite{apple-private-relay}. In the ad ecosystem space, they are part of 
Apple's Safari browser proposal for privacy-preserving click measurements~\cite{safari-clicks}.

The formal definition of security of blind signatures~\cite{JC:PoiSte00,C:JueLubOst97,RSA:AbdNamNev06,JC:SchUnr17} requires two security properties: \emph{blindness} and \emph{one-more unforgeability}. Blindness guarantees that an adversarial signer can neither learn the message in the signing protocol nor link a particular message-signature pair to a protocol execution.  One-more unforgeability guarantees that an adversary cannot produce more signed messages than the number of times it invoked the signing protocol.  

It is important that security hold even as the the blind signing protocol is executed together with other protocols.  At a minimum, therefore, the blind signing protocol needs to be concurrently self-composable.
Unfortunately, when executed concurrently, some otherwise attractive blind signing protocols~\cite{C:Okamoto92,ICICS:Schnorr01,C:AbeOka00,C:Brands93,paquin2013u-prove,CCS:BalLys13,SP:STVWJG16,cryptoeprint:2017:682,JC:GJKR07} are no longer one-more-unforgeable; not in the sense that their proofs of security no longer apply, but recently a concrete and practical attack was discovered~\cite{EC:BLLOR21}. 
A line of work seeking to obtain concurrently secure blind signatures has blossomed recently; PI Lysyanskaya is actively working in this area~\cite{}.\todo{add refs}

\paragraph{Anonymous credentials.} Anonymous credentials~\cite{chaum85,lrsw99,camlys01a,lysyan02a,camlys04} allow Alice to prove to
the access provider that she has a set of credentials, issued by some
trusted issuer, or a set of issuers, that allow her to access the
resource.  What makes them \emph{anonymous} is the fact that Alice's
proof is \emph{zero-knowledge}, which means that the access provider
learns nothing about Alice other than the fact that she possesses the
needed credentials (so in particular, it does not learn who she is, or
in fact any other information that would allow him to link this
transaction to another transaction of the same user).  Moreover, Alice
can also obtain credentials anonymously: an issuer need not know who
Alice is in order to issue a credential.

As a result of decades of research, it has been demonstrated that \emph{everything that can be done with non-anonymous credentials can also be done with anonymous credentials}.  Specifically, there are anonymous credential
systems that are provably secure~\cite{lrsw99,lysyan02a,camlys02b},
efficient enough for use in practice even on severely constrained
devices~\cite{bhrlpb12,CCS:BalLys13}, allow issuers to place limits on how many
times and in what context credentials can be
used~\cite{caholy05,chklm06}, make it possible to revoke anonymous
credentials essentially as effectively as non-anonymous digital
credentials~\cite{camlys02a,lipeyu12a,lipeyu12b}, and yield themselves
to identity escrow add-ons, which make it possible for a trusted
anonymity revocation trustee, or a set of trustees, to find out
Alice's identity in an emergency, after the transaction took
place~\cite{bacaly04}.  These results have attracted wide attention
beyond the cryptographic community: they have been implemented by
industry leaders such as IBM and Microsoft, have found their way into
industrial standards (such as the TCG standard), and underlined the
policies that both the United States government and the EU government
have towards balancing privacy and legitimate identification and
authentication needs.  

Anonymous credentials can be seen as a powerful generalization of blind signatures.  In obtaining a credential, not only does a user now have in their possession a token that is unlinkable to the transaction in which it was issued, but, importantly, this token now has some attributes that are certified by the issuer.  Additionally, this token can be used, unlinkably, more than once --- how many times it can be shown depends on the parameters of the overall system.  Thus, concurrent composition of anonymous credentials is also difficult to achieve; known provably secure systems only show security in the sequential setting~\cite{} or by resorting to cumbersome and hard-to-set-up tools~\cite{}. \todo{add refs}

%\todo{Anna to fill in some background.}



%
%

%\subsection{Preventing abuse when paying for ads, the privacy-preserving way}
%\label{r:fraud}

%\subsubsection{The big picture}




%\paragraph{Motivation.}

%\paragraph{Background.}

%\paragraph{Vision.}

%\paragraph{Challenges.}

%\paragraph{Approach.}

\subsection{Preventing sybils with anonymous credentials} 
%A better approach is when a platform uses anonymous credentials to tell sybils apart from legitimate users.
%
Suppose that the platform issues credentials to users who have performed some actions (such as staying on the platform, interacting with it) that increase the platform's trust that these users are humans rather than sybils.  
%
The more such credentials a user's browser has collected, the more evidence the platform has that the user is not a sybil.  

Anonymous credentials~\cite{} can be issued and shown anonymously.
% 
In other words, the issuing party (here, the platform) does not learn anything about the recipient of the credential.
%
A user can subsequently prove possession of his credential(s), without revealing any additional information, and in a way that cannot be linked to the transaction in which the credential(s) was (were) issued.
%
It is possible to limit the number of times an anonymous credential can be shown~\cite{}, so that an adversary cannot perform a sybil attack by simply disseminating the keys necessary for showing these credentials.

\paragraph{First step: honest-but-curious platform.} Building on existing anonymous credential schemes, we will design a practical end-to-end anonymous authentication protocol. On a high level, we will show how to ensure that a user is not a sybil without revealing their identity to the authentication provider. We assume an identity provider (e.g., Meta) who can be trusted to run infrastructure and to correctly execute the protocols (i.e., honest-but-curious) and users who authenticate from one or more client devices.

Our general approach is based on \emph{anonymous authentication tokens} (AATs) that the authentication provider and clients jointly generate, and which a verifier (who in some settings may be the authentication provider) can verify as genuine without learning the client’s identity. To protect against replay attacks and to limit the usefulness of stolen authentication tokens, each token has a unique serial number and expires after a fixed time. The (single-factor) authentication flow is as follows:
\begin{enumerate}[nosep]
\item The authentication provider orchestrates the protocol for picking the trusted parameters of the system, needed~\cite{lr22} to ensure that the overall system remains secure. Therefore the protocol for picking these parameters must be secure and include trustworthy, non-colluding entities (e.g., EFF, ACLU, other companies).
\item Clients periodically interact with the authentication provider to obtain the credentials they need to generate anonymous authentication tokens. AAT generation might be tied to existing proofs of a non-sybil account with the authentication provider (e.g., some number of friends or established account activity). The underlying cryptographic algorithm may be full-blown anonymous credentials~\cite{EC:camlys01} or anonymous credentials light~\cite{CCS:ballys13}.
\item When a client authenticates to a third-party site, they supply a fresh AAT to the site, which sends it to the authentication provider, who validates it (without learning the client’s identity). If the validation passes, the authentication provider informs the site that a valid user authenticated.
\end{enumerate}

%\paragraph{Challenges.} 
Making this work requires solving both cryptographic challenges and systems ones; here we list some of the research problems that need to be solved:
\begin{itemize}[nosep]
\item The design of the cryptographic protocol for setting up the system, which may require a custom multi-party computation protocol.
\item Adapting existing protocols for obtaining and using tokens; this requires making design choices regarding when tokens are issued, when they expire, whether they are generated locally (with multi-use rate-limited credentials~\cite{chklm06}) or pre-loaded (with anonymous credentials light~\cite{CCS:ballys13}).
\item Addressing systems concerns: what happens if the client’s browser storage is erased and they need new keys? How are keys stored in the browser and protected?
\item Scale: How would a daily issuance of fresh authentication tokens to clients work at the scale of Meta?
%\item Maintaining a database of used-up tokens to prevent token replay.
\end{itemize}


\paragraph{Reducing trust in the platform.}
In the scenario above, a malicious platform can create sybils and charge the advertiser for fraudulent ads.  In order to increase the advertiser's confidence that a user is not a sybil, we can have several platforms grant credentials to a user; thus if the advertiser trusts one of them, that's evidence.  However, which platforms have given a user credentials may reveal information about the user's identity.  Thus we want the user to say "I have AATs from 2 out of 3 platforms, won't tell you which."

\paragraph{Preventing sybils with anonymous credentials} 

%What works currently. %Malte

%Explanation of what we propose will happen. %Malte writes up our whiteboard design.

%Can use off-the-shelf anonymous credentials light, but they are not concurrently secure. %Anna

%Or can use concurrently secure blind sigs, but trust the user to encode the right stuff. %Anna

%Can use off-the-shelf heavy-weight anonymous credentials, but they are not as fast as we'd like, and have some features that are not necessary for this application. %Anna

%How all this integrates and gives us something new and cool. %Malte

\subsubsection{Crypto open problems}
%Anna

Two big problems: performance and being able to support attributes in a concurrently secure fashion.  Using known techniques can achieve one or the other (the former under not-so-great cryptographic assumptions).

Make ACL concurrently secure.  AKA Concurrent blind signatures with attributes more efficiently.

Take an existing concurrently secure blind sig, for example blind RSA, and explore adding attributes.

Delegation of blind sig issuance capability.

Post-quantum.

\subsubsection{Systems open problems}

%
Online advertising operates at huge scale, with platforms serving millions of ad impressions per second, served by distributed systems of hundreds or thousands of servers.
%
Any practical solution to privacy-preserving advertising must therefore be \emph{scalable}, but making the proposed designs scalable requires addressing three challenges: (1) scaling to many servers, (2) the per-request costs imposed by cryptography, and (3) handling untrusted client browsers.
%

%
A classic technique for scaling content delivery on the internet to many (geographically distributed) servers is to use a Content Distribution Network (CDN), such as Cloudflare or Akamai.
%
However, a naive solution where each CDN server has access to the cryptographic material needed is insecure. \todo{can't just give Akamai your cryptographic keys, need to certify each Akamai server involved.}
%

%
Adding cryptography to real-time ad auctions and ad serving logic potentially increases the cost of each request by orders of magnitude---a prohibitive cost for the platform operator (who would have to buy and run many more servers) that would make the design impractical.
%
We plan to solve this problem by basing our protocols on lightweight cryptography that can execute quickly on the server side, and by leveraging offline pre-computation of secrets, such as a daily handshake between the ad platform and a client browser that provides a batch of ad tokens generated efficiently or during periods of low load on the platform (\eg at night).
%
%Also dealing with the more expensive crypto operations as part of being a platform/publisher.  Don't want to be orders of magnitude more expensive than today.

%
On the \emph{client side}, we must ensure that the client device follows the protocols correctly.
%
However, client software (such as browsers and device operating systems) are under the user's control and can be modified with malicious intent (\eg by a click farm operator).
%
We will therefore develop attestation mechanisms that ensure that the user's browser is following the correct protocol, \eg by relying on trusted hardware, such as TPMs and TEEs, that is now commonly available.
%
These hardware extensions can produce a cryptographic proof of the integrity of the running executable, and we imagine that an ad platform would keep a centralized allow-list of approved browser releases.
%

%
Finally, the client-side ad delivery needs to be backwards-compatibile: it should use existing mechanisms, such as embedded images, JavaScript, and HTML iFrames, so browsers can integrate it easily.
%
This requires conveying the cryptographic information needed in addition to the actual ad, in a way that works on existing browsers and without using identifying per-user cookies.
%
For this, we will investigate integrating the cryptographic material in the ad's data itself (\eg image metadata, or HTML code served), or adding minimal extensions to the browser's DOM to convey this additional information in a backwards-compatible way.
%


\subsection{Targeting/bidding}
\label{r:matching}

this part is significantly harder.  Assumptions about user interest categories.

Validation schemes between publisher, platform and user.  Publisher tries to match ads to user interests.  Platform verifies (for a small sample of users) the interests as presented by the publisher against the user's browser history, and has a way to score whether the publisher is good at matching users with ads.  (Possibly on a per-campaign basis.)

Calculating the conversion rate: how many people clicked on the ad, and also how many then proceeded to buy.

\begin{enumerate}
\item Send multiple ads to browser
\item Browser decides which ad to show based on locally-computed interests
\item If user clicks on the ad, the advertiser finds out
\item To charge correctly, the platform needs to find out, so we need either (1) an incentive for the advertiser to communicate with the platform, or (2) notify the platform about the conversion
\item Multiple possible solutions: advertiser needs to pass conversion token to platform to get information it needs in order to assess effectiveness of campaign; anonymizing proxy between browser and platform to handle conversion notifications
\item All of the messages need to be privacy-preserving as far as the identity of the user is concerned
\end{enumerate}

