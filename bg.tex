\section{Background}
\label{s:bg}

\subsection{Online Advertising}

Today's web ecosystem crucially relies on online advertising as a source of
revenue for web services that are free to end users.
%
Advertising is a trillion-dollar market, with Google---one of the key
players---seeing revenue upwards of \$140B in 2020 alone~\cite{xxx}.
%
Much of online advertising is \emph{targeted}, meaning that different users
see different ads, targeted to their individual interests, as determined
from user behavior, browsing history, and demographic information.
%

%
Most websites and service operators that display ads outsource the choice of
which ads to show to a particular user to a third-party.
%
The ad market therefore involves interaction between three parties:
\begin{enumerate}
 \item the \textbf{advertiser}, who seeks to show ads to end users who
   are likely to engage with them (possibly spending money with the
   advertiser as a result), and who is willing to pay for users to view
   these ads;
 %
 \item the \textbf{publisher}, who operates a website that has an audience
   of users to whom advertisers might wish the advertise, and who seeks
   to monetize their content and audience by showing ads;
 %
 \item the \textbf{platform} or ``ad network'', who connects advertisers
   to publishers who have users with the right interest, and takes cut
   of the publisher's ad revenue; and
 %
 \item the \textbf{end user}, who uses the publisher's website and generates
   behavioral data (which could be sensitive) used for ad targeting in doing so,
   who sees the platform's chosen advertiser's ads and potentially interacts
   with them.
\end{enumerate}
%
Well-known platforms include Google's and Meta's Ad business, as well
as Bing Ads, Criteo, Zedo, and others.
%
As a middle-man between advertisers, publishers, and end users, the platform
holds tremendous power and receives significant sensitive information, including
who viewed which ad and whether they interacted with it.
%

\subsection{Ad Targeting}

%
In a \emph{targeted advertising} setting, a publisher shares information about
their users with the platform, and the platform chooses ads to show to the
users from their inventory of advertiser-provided ads.
%

\begin{itemize}
  \item RTB (real-time bidding)
  \item Need high-fidelity information about user interests
  \item Cross-site trackers, reference Apple privacy change impact
  \item Google FLOC?
\end{itemize}

\subsection{Click Fraud}

%
Advertising involves money changing hands, and thus provides incentives for
fraud.
%
In the online ad ecosystem's threat model, individual parties seek to maximize
their own revenue or minimize their costs.
%
Collusion between parties can happen (\eg a publisher may create sybil end
users), and is common.
%

%
The most important challenge is \textbf{click fraud}, which describes the idea
of creating fake ad impressions and engagements in order to extract revenue from
an advertiser.
%
In particular, a publisher gains financially from showing more ads and from
showing higher-value ads.
%
Consequently, they might generate fake traffic via sybil users (e.g., ``click
farms''~\cite{understanding-ad-fraud}), or misrepresent the real users' interest
towards the platform in order to get given more valuable ads in RTB.
%
\todo{Malte/Anna to insert more detail from FB document.}
%

%
The easiest way to combat click fraud is for the platform to identify users,
collect their information, and monitor their behavior at scale.
%
Knowing more about the user make sybils expensive to create and maintain, but is
of course diametrically opposed to the users' justified desire for online
privacy.
%

\subsection{Related Work}
\label{s:bg-related}

\paragraph{End-user privacy \vs the platform.}
%
PrivAd~\cite{privad} introduces an untrusted, anonymizing proxy between the
client and the platform.
%
Has some click fraud detection; mixes requests from different clients for user
anonymity.
%
Good, practical performance, as doesn't need any heavy crypto.
%

%
ObliviAd~\cite{obliviad} relies on hardware-based private information retrieval
(PIR) for ad distribution and mixing of billing tokens to establish client
privacy versus the platform.
%

%
Adnostic~\cite{adnostic} (Boneh et al.) proposes a browser-based user profiling
solution that moves the ad targeting into the user's browser and hides the
specific ad choice from the platform, but provides a cryptographic protocol for
billing the correct advertisers.
%
Green et al.~\cite{adnostic+} improved scale over Adnostic's protocol.
%

Pri-RTB~\cite{pri-rtb} is a private RTB protocol for a semi-honest setting,
based on additively homomorphic encryption of user interest profiles.
%
It only seeks to hide user identities from advertisers, but trusts the
auctioneer (the platform) to run the computation correctly.
%

%
FLoC~\cite{floc} tracks user interests on the client side, and builds cohorts of
users with similar interests; platform them targets ads to the cohorts rather
than individuals.
%

\paragraph{Publisher/Advertiser trust in the platform.}

%
Verifiable, private auctions; MPC based ad auction work.
%

\paragraph{Click fraud prevention.}
%
\todo{Add some references}
%

\subsection{A better approach: Privacy-First Advertising}

%
In this research, we plan to develop an alternative to today's privacy-invasive
advertising ecosystem.
%
Our starting point is simple: we seek the reduce the information exchanged
between the parties in the ad ecosystem to \emph{the minimum information
necessary for successful ad targeting and for the parties to trust each other}
that they have executed their part of the protocol faithfully and correctly.
%

%
Our approach draws on cryptographic tools (\S\ref{s:bg-crypto}) and some minor
changes to sofware, particular end-user browsers.

\subsection{Blind Signatures and Anonymous Credentials}
\label{s:bg-crypto}

\todo{Anna to fill in some background.}
