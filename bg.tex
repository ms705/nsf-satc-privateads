\section{Background and Problem Areas}
\label{s:bg}

\subsection{Online Advertising}

Today's web ecosystem crucially relies on online advertising as a source of
revenue for web services that are free to end users.
%
Advertising is a trillion-dollar market, with Google---one of the key
players---seeing revenue upwards of \$140B in 2020 alone~\cite{xxx}.
%
Much of online advertising is \emph{targeted}, meaning that different users
see different ads, targeted to their individual interests, as determined
from user behavior, browsing history, and demographic information.
%

%
Most websites and service operators that display ads outsource the choice of
which ads to show to a particular user to a third-party.
%
The ad market therefore involves interaction between three parties:
\begin{enumerate}
 \item the \textbf{advertiser}, who seeks to show ads to end users who
   are likely to engage with them (possibly spending money with the
   advertiser as a result), and who is willing to pay for users to view
   these ads;
 %
 \item the \textbf{publisher}, who operates a website that has an audience
   of users to whom advertisers might wish to advertise, and who seeks
   to monetize their content and audience by showing ads;
 %
 \item the \textbf{platform} or ``ad network'', who connects advertisers
   to publishers who have users with the right interest, and takes a cut
   of the publishers' ad revenues; and
 %
 \item the \textbf{end user}, who uses the publisher's website and generates
   behavioral data (which could be sensitive) used for ad targeting in doing so,
   who sees the platform's chosen advertiser's ads and potentially interacts
   with them.
\end{enumerate}
%
Well-known platforms include Google's and Meta's Ad business, as well
as Bing Ads, Criteo, Zedo, and others.
%
As a middle-man between advertisers, publishers, and end users, the platform
holds tremendous power and receives significant sensitive information, including
who viewed which ad and whether they interacted with it.
%

\subsection{Ad Targeting}

%
In a \emph{targeted advertising} setting, a publisher shares information about
their users with the platform, and the platform chooses ads to show to the
users from their inventory of advertiser-provided ads.
%

\begin{itemize}
  \item RTB (real-time bidding)
  \item Need high-fidelity information about user interests
  \item Cross-site trackers, reference Apple privacy change impact
  \item Google FLOC?
\end{itemize}

\subsection{Paying for Ads and Preventing Abuse}
\todo{there are several types of fraud we want to address:
(1) is the person who clicked on an ad legitimate, not a sybil?  
(2) correctly accounting for which ad led a (legitimate) user to later visit the advertiser's site and/or complete a purchase. (3) making sure that the publisher/platform are rewarded for showing a successful ad.
}

%\todo{break up this section because it both describes the background and proposed research.}

%
Advertising involves money changing hands, and thus provides incentives for
fraud.
%
In the online ad ecosystem's threat model, individual parties seek to maximize
their own revenue or minimize their costs.
%
Collusion between parties can happen (\eg a publisher may create sybil end
users), and is common.
%

%
The most important challenge is \textbf{click fraud}, which describes the idea
of creating fake ad impressions and engagements in order to extract revenue from
an advertiser.
%
In particular, a publisher gains financially from showing more ads and from
showing higher-value ads.
%
Consequently, they might generate fake traffic via sybil users (e.g., ``click
farms''~\cite{understanding-ad-fraud}), or misrepresent the real users' interest
towards the platform in order to get given more valuable ads in RTB.
%
The easiest way to combat click fraud is for the platform to identify users,
collect their information, and monitor their behavior at scale.
%
Knowing more about the user make sybils expensive to create and maintain, but is
of course diametrically opposed to the users' justified desire for online
privacy.
%
Thus, a privacy-preserving ways of preventing sybils is a desirable alternative to this approach.


In ad systems that reward a publisher for showing an ad on which the user acts later --- not by clicking the ad when it's shown, but by visiting the advertised site later and doing something meaningful, like making a purchase --- an even more sinister (from the privacy point of view) approach to tracking proliferated.  
%
An ad puts a cookie in the user's browser; when the user later visits a relevant site, it scans all the cookies for those relevant to its ad campaigns, thereby discovering which ad(s) caused the user to visit the site.
%
These cookies are so damaging to consumer privacy that a push to remove the support for these so-called ``third-party" cookies has been successful\footnote{\url{https://www.consumerreports.org/advertising-marketing/internet-advertising-is-about-to-change-third-party-cookies-a6221885875/}}.
%
Of course, the beneficiaries of this success are large platforms that do not need cookies to track users' activities, and can therefore benefit from the demise of smaller ad networks.
%
Unless a privacy-friendly way to handle the ad economy can be found, these platforms will have an ever greater incentive to track all of our activities.

Finally, in addition to preventing malicious behavior from publishers, such as click fraud, ad ecosystems need to have mechanisms that ensures that advertisers reward publishers and platforms for serving successful ads.  




\subsection{Related Work}
\label{s:bg-related}

\paragraph{End-user privacy \vs the platform.}
%
PrivAd~\cite{privad} introduces an untrusted, anonymizing proxy between the
client and the platform.
%
Has some click fraud detection; mixes requests from different clients for user
anonymity.
%
Good, practical performance, as doesn't need any heavy crypto.
%

%
ObliviAd~\cite{obliviad} relies on hardware-based private information retrieval
(PIR) for ad distribution and mixing of billing tokens to establish client
privacy versus the platform.
%

%
Adnostic~\cite{adnostic} (Boneh et al.) proposes a browser-based user profiling
solution that moves the ad targeting into the user's browser and hides the
specific ad choice from the platform, but provides a cryptographic protocol for
billing the correct advertisers.
%
Green et al.~\cite{adnostic+} improved scale over Adnostic's protocol.
%

Pri-RTB~\cite{pri-rtb} is a private RTB protocol for a semi-honest setting,
based on additively homomorphic encryption of user interest profiles.
%
It only seeks to hide user identities from advertisers, but trusts the
auctioneer (the platform) to run the computation correctly.
%

%
FLoC~\cite{floc} tracks user interests on the client side, and builds cohorts of
users with similar interests; platform them targets ads to the cohorts rather
than individuals.
%

\paragraph{Publisher/Advertiser trust in the platform.}

%
Verifiable, private auctions; MPC based ad auction work.
%

\paragraph{Click fraud prevention.}
%
\todo{Add some references}
%

\subsection{A better approach: Privacy-First Advertising}

%
In this research, we plan to develop an alternative to today's privacy-invasive
advertising ecosystem.
%
Our starting point is simple: we seek the reduce the information exchanged
between the parties in the ad ecosystem to \emph{the minimum information
necessary for successful ad targeting and for the parties to trust each other}
that they have executed their part of the protocol faithfully and correctly.
%

%
Our approach draws on cryptographic tools (\S\ref{s:bg-crypto}) and some minor
changes to software, particular end-user browsers.

\subsection{Blind Signatures and Anonymous Credentials}
\label{s:bg-crypto}

\paragraph{Blind signatures.} A blind signing protocol is a protocol between a signer and a user in which the user outputs a digital signature on the desired message, while the signer learns nothing about the message or the resulting signature.
A blind signature scheme is a signature scheme that has a blind signing protocol.  In spite of having a relatively long history (they were introduced almost forty years ago by David Chaum~\cite{C:Chaum82}), blind signatures are a subject of excitement in the cryptography research community at the moment because they can be used as privacy-preserving authentication tokens that can replace browser cookies in certain applications, for example by the VPN by Google One (\url{https://one.google.com/about/vpn/howitworks}) and Apple's iCloud Private Relay (\url{https://www.apple.com/privacy/docs/iCloud_Private_Relay_Overview_Dec2021.PDF}). In the ad ecosystem space, they are part of 
Apple's Safari browser proposal for privacy-preserving click measurements (\url{https://webkit.org/blog/11940/pcm-click-fraud-prevention-and-attribution-sent-to-advertiser/}).

The formal definition of security of blind signatures~\cite{JC:PoiSte00,C:JueLubOst97,RSA:AbdNamNev06,JC:SchUnr17} requires two security properties: \emph{blindness} and \emph{one-more unforgeability}. Blindness guarantees that an adversarial signer can neither learn the message in the signing protocol nor link a particular message-signature pair to a protocol execution.  One-more unforgeability guarantees that an adversary cannot produce more signed messages than the number of times it invoked the signing protocol.  

It is important that security hold even as the the blind signing protocol is executed together with other protocols.  At a minimum, therefore, the blind signing protocol needs to be concurrently self-composable.
Unfortunately, when executed concurrently, some otherwise attractive blind signing protocols~\cite{C:Okamoto92,ICICS:Schnorr01,C:AbeOka00,C:Brands93,paquin2013u-prove,CCS:BalLys13,SP:STVWJG16,cryptoeprint:2017:682,JC:GJKR07} are no longer one-more-unforgeable; not in the sense that their proofs of security no longer apply, but recently a concrete and practical attack was discovered~\cite{EC:BLLOR21}. 
A line of work seeking to obtain concurrently secure blind signatures has blossomed recently; PI Lysyanskaya is actively working in this area~\cite{}.\todo{add refs}

\paragraph{Anonymous credentials.} Anonymous credentials~\cite{chaum85,lrsw99,camlys01a,lysyan02a,camlys04} allow Alice to prove to
the access provider that she has a set of credentials, issued by some
trusted issuer, or a set of issuers, that allow her to access the
resource.  What makes them \emph{anonymous} is the fact that Alice's
proof is \emph{zero-knowledge}, which means that the access provider
learns nothing about Alice other than the fact that she possesses the
needed credentials (so in particular, it does not learn who she is, or
in fact any other information that would allow him to link this
transaction to another transaction of the same user).  Moreover, Alice
can also obtain credentials anonymously: an issuer need not know who
Alice is in order to issue a credential.

As a result of decades of research, it has been demonstrated that \emph{everything that can be done with non-anonymous credentials can also be done with anonymous credentials}.  Specifically, there are anonymous credential
systems that are provably secure~\cite{lrsw99,lysyan02a,camlys02b},
efficient enough for use in practice even on severely constrained
devices~\cite{bhrlpb12,CCS:BalLys13}, allow issuers to place limits on how many
times and in what context credentials can be
used~\cite{caholy05,chklm06}, make it possible to revoke anonymous
credentials essentially as effectively as non-anonymous digital
credentials~\cite{camlys02a,lipeyu12a,lipeyu12b}, and yield themselves
to identity escrow add-ons, which make it possible for a trusted
anonymity revocation trustee, or a set of trustees, to find out
Alice's identity in an emergency, after the transaction took
place~\cite{bacaly04}.  These results have attracted wide attention
beyond the cryptographic community: they have been implemented by
industry leaders such as IBM and Microsoft, have found their way into
industrial standards (such as the TCG standard), and underlined the
policies that both the United States government and the EU government
have towards balancing privacy and legitimate identification and
authentication needs.  

Anonymous credentials can be seen as a powerful generalization of blind signatures.  In obtaining a credential, not only does a user now have in their possession a token that is unlinkable to the transaction in which it was issued, but, importantly, this token now has some attributes that are certified by the issuer.  Additionally, this token can be used, unlinkably, more than once --- how many times it can be shown depends on the parameters of the overall system.  Thus, concurrent composition of anonymous credentials is also difficult to achieve; known provably secure systems only show security in the sequential setting~\cite{} or by resorting to cumbersome and hard-to-set-up tools~\cite{}. \todo{add refs}

%\todo{Anna to fill in some background.}
