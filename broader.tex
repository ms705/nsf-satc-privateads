\section{Broader Impacts}
\noindent\textbf{Implementation and technology transfer.}  
The idea for this project arose when we had a series of discussions with Facebook (now Meta) engineers and researchers who work on ads and who are interested in moving to a more privacy-preserving model.  Current state of the art proved to be insufficient for their needs, which gave rise to this research agenda.
%
Since then, we received additional indications from software engineers that this was a welcome development.  Google announced held a meeting titled ``Google privacy preserving ads ecosystem workshop" for researchers working in this space in June 2021.  Fastly, Apple and Cloudflare engineers created a proposal for a standardized blind signature scheme~\cite{ietf:djw21,ietf:djw22}.  

%The PI has established  In fact, the 
%Hyperledger developers have communicated their interest in delegatable credentials in general and the PI's work and mercurial signatures in particular.  

The specific activities to ensure that the efforts of this project will be relevant to practitioners include participation in meetings such as the Real World Cryptography (RWC) conference and the HACS workshop at which academic cryptography researchers interact with heads of cryptography groups of leading software companies.  The PIs have established a strong track record of participation at such events: (1) we have given invited talks at RWC \todo{where else?}; (2) PI Lysyanskaya serves on the Board of Directors of the IACR (parent organization for RWC); (3) \todo{more stuff}.
Further, the PIs will collaborate with industry efforts to create open-source community-vetted implementations of our designs with an eye towards applications in practice.  

\noindent\textbf{Diversity.} PI Lysyanskaya has a strong track record of mentoring women. Six of her past Ph.D. advisees were women; she has given talks at events such as the Women in Theory workshop and the Grace Hopper Celebration, and is committed to continuing this work.
PI Schwarzkopf has a track record of focused efforts on improving diversity in computer systems research. In particular, he runs an exploreCSR program focused on encouraging more female students to consider graduate research in computer systems~\cite{explorecsr-systems}. He also actively recruits and mentors HUG undergraduate students in computing for research, and hires a diverse TA staff for his courses.

\noindent\textbf{Education and outreach.} Graduate students will be involved in all aspects of the research activities.  PI Lysyanskaya
regularly teaches \textbf{CSCI1510} which is an undergrad cryptography course, \textbf{CSCI1040} which is a pre-requisites free course on what cryptography can offer the world, and \textbf{CSCI2590}, a graduate topics course on cryptography; results of this work will be incorporated into the content of all three courses.
%
PI Schwarzkopf teaches \textbf{CSCI2390}, a graduate seminar on privacy-conscious computer systems design. The course covers web advertising and one major topic is how to design privacy-conscious systems that still work with the ad-based business model at the core of today's internet. Schwarzkopf will embed ideas from this project in the course in the form of reading assignments and a new assignment on privacy-conscious advertising.
%

\noindent\textit{Brown University's Master in Cybersecurity Program.} While any Bachelor's degree holder with sufficient background can be admitted to this program, it was 
developed with corporate executives in mind. It allows them to work full-time while getting their degree; this is accomplished by augmenting regular advanced courses in our curriculum with an online/remote track.  Several of the courses taught by the PIs are part of this curriculum.  Alumnae include CISOs of Fortune-500 companies and other individuals of national standing.  
The results of this research will inform what we teach them and, as a result, what happens in practice.

\noindent\textit{Public policy.} The PIs are committed to ensuring that public policy makers are informed about how privacy-enhancing technologies can help society.  Our past efforts include op-ed pieces~\cite{projo1,csm,projo2}, direct feedback to policy makers~\cite{annacdt}, and working with public advocacy groups~\cite{epic15}.
