\section{Research}
\label{r:stuff}

%
We plan to tackle the problem of privacy-first advertising in two broad steps:
first, we establish the primitives needed to achieve trust between parties in
the ad ecosystem to avoid click fraud without requiring privaxy-invasive data
collection (\S\ref{r:fraud}), and then we address the ad matching process and
hide users' interests and behavior from the other parties in the ecosystem
(\S\ref{r:matching}).
%

\subsection{Preventing click fraud}
\label{r:fraud}

\subsubsection{What we can do based on existing cryptographic algorithms}
What works currently. %Malte

Explanation of what we propose will happen. %Malte writes up our whiteboard design.

Can use off-the-shelf anonymous credentials light, but they are not concurrently secure. %Anna

Or can use concurrently secure blind sigs, but trust the user to encode the right stuff. %Anna

Can use off-the-shelf heavy-weight anonymous credentials, but they are not as fast as we'd like, and have some features that are not necessary for this application. %Anna

How all this integrates and gives us something new and cool. %Malte

\subsubsection{Crypto open problems}
%Anna

Two big problems: performance and being able to support attributes in a concurrently secure fashion.  Using known techniques can achieve one or the other (the former under not-so-great cryptographic assumptions).

Make ACL concurrently secure.  AKA Concurrent blind signatures with attributes more efficiently.

Take an existing concurrently secure blind sig, for example blind RSA, and explore adding attributes.

Delegation of blind sig issuance capability.

Post-quantum.

\subsubsection{Systems open problems}
% Malte
How to scale using CDNs -- can't just give Akamai your cryptographic keys, need to certify each Akamai server involved. Also dealing with the more expensive crypto operations as part of being a platform/publisher.  Don't want to be orders of magnitude more expensive than today.

On the user side: makes sure that the user's browser is following the correct protocol.

Backwards-compatibility.  Ad delivery should work using existing mechanisms, so no need for a new way to display ads in the browser.
Convey the crypto information in addition to the actual ad, in a way that would work on existing browsers, and without cookies, or at least without identifying cookies.



\subsection{Targeting/bidding}
\label{r:matching}

this part is significantly harder.  Assumptions about user interest categories.

Validation schemes between publisher, platform and user.  Publisher tries to match ads to user interests.  Platform verifies (for a small sample of users) the interests as presented by the publisher against the user's browser history, and has a way to score whether the publisher is good at matching users with ads.  (Possibly on a per-campaign basis.)

Calculating the conversion rate: how many people clicked on the ad, and also how many then proceeded to buy.
