\section{Research}
\label{r:stuff}

%
We plan to tackle the problem of privacy-first advertising in two broad steps:
first, we establish the primitives needed to carry out financial accounting in
the ad ecosystem without requiring privacy-invasive data
collection (\S\ref{r:fraud}), and then we address the ad matching process and
hide users' interests and behavior from the other parties in the ecosystem
(\S\ref{r:matching}).
%

\subsection{Preventing abuse when paying for ads, the privacy-preserving way}
\label{r:fraud}

\subsubsection{The big picture}
The ad market involves interaction between four types of mutually distrustful parties:
\begin{enumerate}
 \item the \textbf{advertiser}, who seeks to show ads to end users who
   are likely to engage with them (possibly spending money with the
   advertiser as a result), and who is willing to pay for users to view
   these ads;
 %
 \item the \textbf{publisher}, who operates a website that has an audience
   of users to whom advertisers might wish to advertise, and who seeks
   to monetize their content and audience by showing ads;
 %
 \item the \textbf{platform} or ``ad network'', who connects advertisers
   to publishers who have users with the right interest, and takes a cut
   of the publishers' ad revenues; and
 %
 \item the \textbf{end user}, who uses the publisher's website and generates
   behavioral data (which could be sensitive) used for ad targeting in doing so,
   who sees the platform's chosen advertiser's ads and potentially interacts
   with them.
\end{enumerate}

The money here flows from the advertisers to the platform(s) for placing ads with publishers so that they will ultimately be seen and acted upon by users.  The advertisers don't entirely trust the platforms to bill them correctly and need accounting mechanisms they can rely on and audit.  Similarly, the publishers do not entirely trust the platforms to accurately reward them for showing ads to visitors to their sites, and want trustworthy and auditable accounting mechanisms as well.  Our goal is to realize these mechanisms without any need for any of these parties to track the users.  

In particular, we want to achieve the following privacy guarantees:
\begin{enumerate}[nosep]
  \item the user is anonymous to the platform at all times;
  \item the publisher is anonymous towards the advertiser;
  \item the advertiser is anonymous towards the publisher;
  \item neither publisher, nor advertiser or platform find out which \emph{specific} interaction (content--ad--user combination)
	resulted in a conversion, only that a real user, a real ad, and a real conversion event were involved.
\end{enumerate}
%
However, we do \emph{not} assume anonymity between users and publishers (since users often authenticate to publishers to use their services), or between users and advertisers if a conversion happens (since the conversion involves a purchase, the advertiser finds out the user's identity anyway).
%

\paragraph{Threat model.}
%
We assume that everyone trusts the platform to be honest-but-curious (\ie faithfully execute our protocol). 
%
We also assume that the platform trusts the client-side browser software to behave correctly.
%
Major ad platforms (\eg Google) in some cases also control browser software; however, since web browsers are open-source software, client-side modification is a possibility.
%
We assume that the ad platform can attest to the browser's integrity using technical means for remote attestation, such as a Trusted Platform Module (TPM).
%
When it encounters an unknown browser, the platform treats it as suspicious (\ie it might be a click farm) and refuses to serve ads to it.
%


%
When a user lands on an advertiser's website, the advertiser wants two pieces of information: (1) is this a real human, or a sybil? and (2) which ad was effective at sending this user here?
%
In case this is a real user, the advertiser needs to reward the publisher/platform who showed the ad to the user.
%
In case the user makes a purchase, the advertiser needs to pay the platform, which rewards the publisher for the success of the ad.
%

%
However, users would prefer to avoid revealing their activity related to ads to the publisher and to the platform.
%

\paragraph{Design overview.}
%
Below, we sketch our privacy-preserving ad ecosystem in which the cryptographic protocols combined with the participants' incentives prevent them from cheating.
%


\begin{itemize}
% \item browser is trusted by platform, and open-source, so auditable
 \item platform issues token to browser when showing ad
 \item browser forwards token to advertiser
 \item at point of conversion/purchase, advertiser issues token back to browser
 \item browser sends token to platform
 \item platform responds with IOU token
 \item browser sends IOU to publisher
\end{itemize}


Next, we describe our research agenda to make this protocol and the big picture sketched in this section a reality.

\subsubsection{Research Goal 1: Efficient Composable Secure Blind Signatures and Other Cryptographic Tokens}

All the tokens generated in our vision of the ad economy need to be untraceable.  In other words, the transaction in which the token is issued, needs to be unlinkable to the one in which it is ``spent," or shown to a verifier.  The most well-studied cryptographic mechanism for such unlinkable tokens is a blind signature scheme.  

\paragraph{Concurrently composable blind signature schemes.} The most efficient concurrently secure blind signatures known is the one due to Bellare et al.~\cite{} under the one-more-RSA assumption.  Unfortunately, the one-more-RSA assumption is somewhat non-standard.  

Other techniques include PI Lysyanskaya's recent work on a transformation that converts a certain class of blind signature schemes~\cite{} that can tolerate $O(\log n)$ concurrent sessions into schemes that can tolerate $O(n)$ concurrent sessions, but at cost of an $O(n)$-fold increase in computation and an $O(\log n)$-fold increase in communication.  

On the negative side, it was recently shown that a broad class of blind signature schemes are not concurrently secure~\cite{}.

Problems to work on:

\begin{enumerate}
\item A better general compiler from somewhat-concurrently secure blind sigs to fully concurrently secure.  The known compilers are based on the user revealing all of his randomness to prove honest behavior.  The chances that he sent badly formed messages and was not caught is $1/n$.  But what if we use aggregation somehow, he reveals one half of what he is doing?  


\item More concurrently secure schemes (discuss Abe's scheme).  Discuss scheme from the LRSW assumption and from linear-DH assumptions.
\item 
\end{enumerate}


\subsubsection{Research Goal 2: Systems Design for Private Advertising}

\subsubsection{Research Goal 3: Scalable Private Advertising}

\subsubsection{Preventing sybils with anonymous credentials} 
%A better approach is when a platform uses anonymous credentials to tell sybils apart from legitimate users.
%
Suppose that the platform issues credentials to users who have performed some actions (such as staying on the platform, interacting with it) that increase the platform's trust that these users are humans rather than sybils.  
%
The more such credentials a user's browser has collected, the more evidence the platform has that the user is not a sybil.  

Anonymous credentials~\cite{} can be issued and shown anonymously.
% 
In other words, the issuing party (here, the platform) does not learn anything about the recipient of the credential.
%
A user can subsequently prove possession of his credential(s), without revealing any additional information, and in a way that cannot be linked to the transaction in which the credential(s) was (were) issued.
%
It is possible to limit the number of times an anonymous credential can be shown~\cite{}, so that an adversary cannot perform a sybil attack by simply disseminating the keys necessary for showing these credentials.

\paragraph{First step: honest-but-curious platform.} Building on existing anonymous credential schemes, we will design a practical end-to-end anonymous authentication protocol. On a high level, we will show how to ensure that a user is not a sybil without revealing their identity to the authentication provider. We assume an identity provider (e.g., Meta) who can be trusted to run infrastructure and to correctly execute the protocols (i.e., honest-but-curious) and users who authenticate from one or more client devices.

Our general approach is based on \emph{anonymous authentication tokens} (AATs) that the authentication provider and clients jointly generate, and which a verifier (who in some settings may be the authentication provider) can verify as genuine without learning the client’s identity. To protect against replay attacks and to limit the usefulness of stolen authentication tokens, each token has a unique serial number and expires after a fixed time. The (single-factor) authentication flow is as follows:
\begin{enumerate}[nosep]
\item The authentication provider orchestrates the protocol for picking the trusted parameters of the system, needed~\cite{lr22} to ensure that the overall system remains secure. Therefore the protocol for picking these parameters must be secure and include trustworthy, non-colluding entities (e.g., EFF, ACLU, other companies).
\item Clients periodically interact with the authentication provider to obtain the credentials they need to generate anonymous authentication tokens. AAT generation might be tied to existing proofs of a non-sybil account with the authentication provider (e.g., some number of friends or established account activity). The underlying cryptographic algorithm may be full-blown anonymous credentials~\cite{EC:camlys01} or anonymous credentials light~\cite{CCS:ballys13}.
\item When a client authenticates to a third-party site, they supply a fresh AAT to the site, which sends it to the authentication provider, who validates it (without learning the client’s identity). If the validation passes, the authentication provider informs the site that a valid user authenticated.
\end{enumerate}

%\paragraph{Challenges.} 
Making this work requires solving both cryptographic challenges and systems ones; here we list some of the research problems that need to be solved:
\begin{itemize}[nosep]
\item The design of the cryptographic protocol for setting up the system, which may require a custom multi-party computation protocol.
\item Adapting existing protocols for obtaining and using tokens; this requires making design choices regarding when tokens are issued, when they expire, whether they are generated locally (with multi-use rate-limited credentials~\cite{chklm06}) or pre-loaded (with anonymous credentials light~\cite{CCS:ballys13}).
\item Addressing systems concerns: what happens if the client’s browser storage is erased and they need new keys? How are keys stored in the browser and protected?
\item Scale: How would a daily issuance of fresh authentication tokens to clients work at the scale of Meta?
%\item Maintaining a database of used-up tokens to prevent token replay.
\end{itemize}


\paragraph{Reducing trust in the platform.}
In the scenario above, a malicious platform can create sybils and charge the advertiser for fraudulent ads.  In order to increase the advertiser's confidence that a user is not a sybil, we can have several platforms grant credentials to a user; thus if the advertiser trusts one of them, that's evidence.  However, which platforms have given a user credentials may reveal information about the user's identity.  Thus we want the user to say "I have AATs from 2 out of 3 platforms, won't tell you which."

\paragraph{Preventing sybils with anonymous credentials} 

%What works currently. %Malte

%Explanation of what we propose will happen. %Malte writes up our whiteboard design.

%Can use off-the-shelf anonymous credentials light, but they are not concurrently secure. %Anna

%Or can use concurrently secure blind sigs, but trust the user to encode the right stuff. %Anna

%Can use off-the-shelf heavy-weight anonymous credentials, but they are not as fast as we'd like, and have some features that are not necessary for this application. %Anna

%How all this integrates and gives us something new and cool. %Malte

\subsubsection{Crypto open problems}
%Anna

Two big problems: performance and being able to support attributes in a concurrently secure fashion.  Using known techniques can achieve one or the other (the former under not-so-great cryptographic assumptions).

Make ACL concurrently secure.  AKA Concurrent blind signatures with attributes more efficiently.

Take an existing concurrently secure blind sig, for example blind RSA, and explore adding attributes.

Delegation of blind sig issuance capability.

Post-quantum.

\subsubsection{Systems open problems}

%
Online advertising operates at huge scale, with platforms serving millions of ad impressions per second, served by distributed systems of hundreds or thousands of servers.
%
Any practical solution to privacy-preserving advertising must therefore be \emph{scalable}, but making the proposed designs scalable requires addressing three challenges: (1) scaling to many servers, (2) the per-request costs imposed by cryptography, and (3) handling untrusted client browsers.
%

%
A classic technique for scaling content delivery on the internet to many (geographically distributed) servers is to use a Content Distribution Network (CDN), such as Cloudflare or Akamai.
%
However, a naive solution where each CDN server has access to the cryptographic material needed is insecure. \todo{can't just give Akamai your cryptographic keys, need to certify each Akamai server involved.}
%

%
Adding cryptography to real-time ad auctions and ad serving logic potentially increases the cost of each request by orders of magnitude---a prohibitive cost for the platform operator (who would have to buy and run many more servers) that would make the design impractical.
%
We plan to solve this problem by basing our protocols on lightweight cryptography that can execute quickly on the server side, and by leveraging offline pre-computation of secrets, such as a daily handshake between the ad platform and a client browser that provides a batch of ad tokens generated efficiently or during periods of low load on the platform (\eg at night).
%
%Also dealing with the more expensive crypto operations as part of being a platform/publisher.  Don't want to be orders of magnitude more expensive than today.

%
On the \emph{client side}, we must ensure that the client device follows the protocols correctly.
%
However, client software (such as browsers and device operating systems) are under the user's control and can be modified with malicious intent (\eg by a click farm operator).
%
We will therefore develop attestation mechanisms that ensure that the user's browser is following the correct protocol, \eg by relying on trusted hardware, such as TPMs and TEEs, that is now commonly available.
%
These hardware extensions can produce a cryptographic proof of the integrity of the running executable, and we imagine that an ad platform would keep a centralized allow-list of approved browser releases.
%

%
Finally, the client-side ad delivery needs to be backwards-compatibile: it should use existing mechanisms, such as embedded images, JavaScript, and HTML iFrames, so browsers can integrate it easily.
%
This requires conveying the cryptographic information needed in addition to the actual ad, in a way that works on existing browsers and without using identifying per-user cookies.
%
For this, we will investigate integrating the cryptographic material in the ad's data itself (\eg image metadata, or HTML code served), or adding minimal extensions to the browser's DOM to convey this additional information in a backwards-compatible way.
%


\subsection{Targeting/bidding}
\label{r:matching}

this part is significantly harder.  Assumptions about user interest categories.

Validation schemes between publisher, platform and user.  Publisher tries to match ads to user interests.  Platform verifies (for a small sample of users) the interests as presented by the publisher against the user's browser history, and has a way to score whether the publisher is good at matching users with ads.  (Possibly on a per-campaign basis.)

Calculating the conversion rate: how many people clicked on the ad, and also how many then proceeded to buy.

\begin{enumerate}
\item Send multiple ads to browser
\item Browser decides which ad to show based on locally-computed interests
\item If user clicks on the ad, the advertiser finds out
\item To charge correctly, the platform needs to find out, so we need either (1) an incentive for the advertiser to communicate with the platform, or (2) notify the platform about the conversion
\item Multiple possible solutions: advertiser needs to pass conversion token to platform to get information it needs in order to assess effectiveness of campaign; anonymizing proxy between browser and platform to handle conversion notifications
\item All of the messages need to be privacy-preserving as far as the identity of the user is concerned
\end{enumerate}
