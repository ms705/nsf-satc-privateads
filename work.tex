\section{Collaboration and Work Plan}

\paragraph{Collaboration.}
%
The PIs are in the same department, and bring complementary expertise on cryptography and secure systems design.
%
In this proposal, Lysyanskaya contributes the primary expertise on blind signatures and cryptography, and will lead the work on developing the new compact token and concurrently-secure anonymous credential schemes.
%
Schwarzkopf provides expertise on real-world distributed systems design, systems security, and the necessary efficiency for operation at scale.
%
He will lead the work on integrating the cryptographic primitives and protocols developed with server-side and client-side software; in particular, this will involve client browser modifications of an open-source browser (Mozilla Firefox or Chromium).
%
Both PIs will co-advise PhD students who work on this project, and jointly teach them how to design provably secure cryptographic primitives (Lysyanskaya taking the lead) and how to build scalable systems infrastructure (Schwarzkopf taking the lead).
%
The Brown CS graduate program is well suited towards this cross-area co-advising.
%
Schwarzkopf has experience doing systems work with students with a theory background (\eg having recently published a paper on GDPR compliance~\cite{gdprizer} with Seny Kamara's students), and Lysyanskaya has extensive experience supervising cryptography implementation projects~\cite{mekhl10,bhrlpb12,hzblpb13,bcdlrsy17}. 
%

\paragraph{Work Plan.}
%
We expect to proceed according to the following timeline.
%
\textbf{Year 1:} Develop first end-to-end system prototype based on existing cryptographic techniques (prototype 0) to validate the system design. Concurrently, we will develop the new cryptographic primitives needed.
%
\textbf{Year 2:} We will prove the security properties of the new cryptographic primitives developed, and integrate these primitives with our second implementation (prototype 1).
%
\textbf{Year 3:} Focus on improving efficiency of the blind signature schemes implemented, extend our prototype to support realistic at-scale measurements, and publish the work in a peer-reviewed venue and open-source the software created.
%
